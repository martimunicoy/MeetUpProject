\section{Data Life Cycle}\label{sec:datalifecycle}

%\smartdiagram[circular diagram:clockwise]{Creation, Storage, Use, Sharing, Archiving, Destruction}

In the following section we have a closer look at the terms "data life cycle" and "data life cycle management".

"Data life cycle" is a term which tries to describe the life of data and information. Since there is no unique definition of a data life cycle, we define the folloing six stages: creation, storage, use, sharing, archiving and destruction  \cite{marburg}, \cite{spirion}.

\begin{enumerate}
	\item \textbf{Creation}: data is created. It can be created as a structured or unstructured set of data, can be obtained by collection, measurements, generated or gathered. We create our data by using the Meetup-API. By performing a web scraping at \url{Wikipedia.org} we gather the number of population, the population density and the land area for different cities and their districts. In this way, we request and create the data which is going to be needed for this project.
	\item \textbf{Storage}: once the data is created, it has to be stored. Depending on where we want to apply this data for, there are different and optimal kinds of storage solutions. However, this is not subject of this report. In our case, both, the data gathered from \url{Meetup.com} and \url{Wikipedia.org}, are saved as csv-files on the hard drive.
	\item \textbf{Use}: data can be viewed, modified analyzed, corrected, interpreted, visualized, joined,... We use our data to visualize the social activity of cities. We join our data from different sources and visualize it on a given map which can be seen as data as well. This data visualization and its further treatment can be used to obtain new data such as the measurement of the social activity per district introduced in section \ref{sec:summaryandgoals}.
	\item \textbf{Sharing}: data can be shared. There are several ways of sharing by using different file formats, applications or operating environments. In this work, we used csv-files and GeoJSON-files to share data from one method to another. For instance, the district coordinates are read from a GeoJSON-file, the Meetup events and the population data from a csv-file. Then, these are the files created by the scripts to share data along different methods.
	\item \textbf{Archiving}: when the data leaves the active use, one should archive it in a suitable way. Since archiving is very similar to storaging, one can use the same methods. One difference may be that saved data needs to be compressed to save storage or protected to prevent unwanted changed. Therefore, the access is slightly more difficult than for data in active use. The file formats used in this project does not satisfy this statement. This is due to the fact that the file formats are chosen to ease the access to their data rather than to save storage space. If we want to keep this data for a long time, we should think on developing a method to compress this data.
	\item \textbf{Destruction}:  there is a last stage in the data life cycle. There are several reasons why one should destroy data. On one hand, the volume of archived data increases and one needs to save storage to store new data. On the other hand, data gets old and could be not needed anymore. Or, if it is not up to date, it can be replaced by newer versions. The second case is applicable for our work. During every single run of the program new csv-files with event data and new csv-files containing populations data are created. Especially the event data underlie a rapid change.
\end{enumerate}

The term "data life cycle management" describes the process that helps to organize the flow of data throughout its life cycle.