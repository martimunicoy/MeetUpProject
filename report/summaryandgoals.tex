\section{Summary and Goals}\label{sec:summaryandgoals}

The human being is social. Every day there are millions of events. Thousands of them are posted at the webpage \url{Meetup.com}. It is a website where people can seek and find other people with the same interest. It helps to organize events and makes them public.

This is the reason why we chose \url{Meetup.com} to analyze the behavior and the preferences of cities and its inhabitants all over the world. Almost all \url{Meetup.com} activities contain information about their location, expressed in terms of latitude and longitude. Thus, by using the Meetup-API, we can request and extract a dataset "latitude-longitude" for all the available \url{Meetup.com} events. Then, we can use this dataset to plot all these located activities on a \url{GoogleMaps.com} map. Our aim here is to create and visualize a density map, a so-called heatmap, where one can see all the activities for a selected city.

Furthermore, the Meetup-API allows the user to get activities arranged by category. We want to use this tool to consider separately different types of events like "Food \& Drink" or "Sports" and study their relative predominance in different cities. By combining this information with the geographical and/or political structure of a city, one may can also identify trending districts and neighborhoods in a city. To do so we make use of GeoJSON-files which can contain the polygonal shape of a district and its name. There are several websites where one can get GeoJSON data or create own data. By using the information from these websites, a GeoJSON dataset is created with all the coordinates of the districts for all the studied cities. This constitutes the third data source that is used in this work.

A fourth data source is \url{Wikipedia.org} where a web scraping script is applied on to extract some useful population data such as the population number, the population density or the land area of each district. With all this information, one can use the following factor to evaluate the social activity per district. This factor is given as

\begin{align*}
	\text{Social activity} = \frac{\text{\# events}}{\text{\# person}}.
\end{align*}

Once downloaded, we can access to our data offline. This is due to the fact that we store it locally by using csv-files (comma-separated files).

This report is structured as follows in \ref{sec:limitationsandfuturework} sections. Firstly, in section \ref{sec:datalifecycle} we discuss the data life-cycle of our data. In section  \ref{sec:toolsanddata}, we present both, the third-party tools that we used and the custom methods that we developed in our own, to achieve this project. They mainly are developed by using Python.  Afterwards, we present our results in section \ref{sec:results} and, in section \ref{sec:legalandethicalissues}, we make a legal and ethical disquisition. Finally, in section \ref{sec:limitationsandfuturework} we discuss the limitations of our work and make proposals for future work.