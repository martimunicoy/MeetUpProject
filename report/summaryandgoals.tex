\section{Summary and Goals}\label{sec:summaryandgoals}

The human being is social. Every day there are millions of events. Some thousand of them are posted at the webpage \url{Meetup.com}. It is a website where people can seek and find other people with the same interest. It helps to organize your events and makes them public. For this reason we can use \url{Meetup.com} to analyze the behavior and the preferences of cities and its inhabitants all over the world since almost all events posted on \url{Meetup.com} have a place which can be expressed as latitude and longitude. By using the \textcolor{red}{request-API} we extract the dataset "latitude-longitude" from \url{Meetup.com}and plot it on a map given by Google Maps. In this way we create a density map, a so-called heatmap, where on can see all the acitivities in the selected city.\\Furthermore, we also consider different types of events like "food \& drinks" of "sports". Combining these information with the geographical and/or political structure of a city one can identify trending districts and neighborhoods in a city. To do so we make use of geojson-files which can contain the polygonal shape of a district and its name. There are several websites where one can get geojson data or create his own data.\\A third data source is \url{Wikipedia.org} where we performe a web scrapping to extract the population number of a city and, \textcolor{red}{if available}, the polulation of a district. Having this information we create a factor to evaluate the \textcolor{red}{degree of acitivity}. This factor is given as

\begin{align*}
	\text{degree of activity} = \frac{\text{events}}{\text{person}}.
\end{align*}

Our data is available offline and online.\\This work is structured as follows: In section \ref{sec:datalifecycle} we discuss the data life-cycle of our data. In the section \ref{sec:toolsanddata} we present our used methods and have a closer look at \textcolor{red}{Gmaps-API, Python and Jupyther, geojson and the webscrapping.} Afterwards, in section \ref{sec:results} our results are presented. In section \ref{sec:legalandethicalissues}, we discuss legal and ethical issues. Finally, in section \ref{sec:limitationsandfuturework} we discuss the limitations of our work and make proposals for further work.