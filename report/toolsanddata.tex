\section{Data Tools}\label{sec:toolsanddata}
\subsection{Python Packages} 
The main work for our project is done in the Python programming language in version 3.6 \cite{python}. This programming language is used to develop different packages. Each package is in charge of managing a specific dataset and it does a certain job. In this section, we are going to introduce all the packages that we have developed and we use to retrieve the data of this project.

\subsection{Meetup Package}
This package has the main purpose of retrieve the information of all the available events in a city. Its main module is called \textit{mu_requests.py}. This module makes use of the \textit{Python 3.6 Requests} package to submit queries to Meetup. It is able to get information about all the available categories and events. Then, it can also parse the data coming from the Meetup Client, whose format is expressed in JSON, and it can store it in a local drive by writing a custom csv file. For instance, all the events found in a city they are written in this way \textit{./csv/name_of_the_city.csv}. This custom csv file contains all the events of a city arranged by category.

\subsection{Mapping Package}
This is the package responsible of plotting all the located events on a map. To do this, first, it needs to read all the events data from the previous csv-file. Then, it uses the read coordinates to display events on a Google Maps map. This functionality uses the \textit{gmaps} library along with a \textit{Jupyter} notebook which allow us to display data on a map.

It has some tools that allows us manage which events we want to display. For example, it allows us to plot only those events that are scheduled in a certain time range or those events which belong to a specific category. We can also control the way the script colors the event marks or their opacity.

Another interesting tool that this package offers is to plot the districts of a city. In this case, GEOJSON files with the delimiting points for each district are required. If you want to plot them according to their population, population data for each district is also required. The script read these files and parses the data and it can plot and paint the districts according to the number of events per capita that each district has.

\subsection{Scraping Package}
This package is involved on retrieving data from \textit{Wikipedia}. Particularly, it searches for the pages belonging to the districts of a city to retrieve information about their population from one of their tables. This process is usually known as web scraping. To check for the appropriate tables, it uses \textit{BeautifulSoup} Package that is an html parser.

Then, the script is designed to look for these tables in an intelligent way.  So, it is able to check different Wikipedia addresses to get the right information.  Firstly, it looks for the English articles but, in some cases, best articles are written in other languages rather than English. So, it is also able to check pages written in Spanish and Deutsch.

Once it gets the right population table it comes the hardest part. It needs to move around the table to look for the proper cells. The functions that are used in the parsing process are able to work with tables, words and numbers expressed in different formats. For instance, it supports tables with joined cells or which arrange their information in different ways.

\subsection{Plotting Package}
The last developed Package is used to plot data from all these retrieved events. It is useful to view and compare the amount of events for each category or city. This script makes use of \textit{pygal package} to create these plots. 

\subsection{Data}

As already noted we used the data sources that are summarized below:

\begin{itemize}
\item Meetup API: To get the events in the examined cities. 
\item Wikipedia: For obtaining data about the population density of the cities and their districts. 
\item Various resources for GeoJSON files: To visualize districts onto Google Maps. \textcolor{red}{TODO}
\item Google Maps: As a general back-end for our visualization.  
\end{itemize}