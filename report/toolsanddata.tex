\section{Tools and Data}\label{sec:toolsanddata}
\subsection{Tools} 
The main work for our project was done in the Python programming language in version 3.6 \cite{python}. The most notable used Python libraries where requests \cite{requests}, gmaps \cite{gmaps}, Jupyter \cite{jupyter} and matplotlib \cite{matplotlib}. The requests package was used to call and request data from the Meetup API \cite{meetupapi}. 

With Jupyter Notebook and gmaps we were able to plot the obtained data from the Meetup API, Wikipedia and other resources in a browser. In detail Jupyter provides the general browser session functionality whereas gmaps extends Jupyter with an interactive map provided by Google Maps \cite{googlemaps}. One can visualize specific points or areas in various colors and densities to create meaningful maps. 
Matplotlib was used to transform the data from population density into specific color schemes. 

\subsection{Data}

As already noted we used the following data sources: 

\begin{itemize}
\item Meetup API: To get the events in the examined cities. 
\item Wikipedia: For obtaining data about the population density of the cities and their districts. 
\item Various resources for GeoJSON files: To visualize districts onto Google Maps. \textcolor{red}{TODO}
\item Google Maps: As a general back-end for our visualization.  

\end{itemize}